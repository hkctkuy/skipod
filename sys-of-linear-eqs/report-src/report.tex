\documentclass[oneside,senior,etd]{BYUPhys}

\usepackage{cmap} % Для корректной кодировки в pdf
\usepackage[utf8]{inputenc}
\usepackage{rotating}

\usepackage[russian]{babel}
\usepackage{amsfonts} % Пакеты для математических символов и теорем
\usepackage{amstext}
\usepackage{amssymb}
\usepackage{amsthm}
\usepackage{graphicx} % Пакеты для вставки графики
\usepackage{subfig}
\usepackage{color}
\usepackage[unicode]{hyperref}
\usepackage[nottoc]{tocbibind} % Для того, чтобы список литературы отображался в оглавлении
\usepackage{verbatim} % Для вставок заранее подготовленного текста в режиме as-is
\usepackage{listings}

\newcommand{\sectionbreak}{\clearpage} % Раздел с новой станицы

\usepackage{tikz}
\usepackage{pgfplots}
\usetikzlibrary{arrows,positioning}
\usepackage{adjustbox}

\usepackage{makecell}
\usepackage{booktabs}
\usepackage{boldline}

\usepackage{xcolor}
\usepackage{soul}
\usepackage{url}
\usepackage{multirow}
\usepackage{amsmath}

\usepackage{pifont}
\usepackage{indentfirst} % Делать отступ в начале первого параграфа

\usepackage{minted}

\usepackage[inline]{enumitem}

\renewcommand
\listingscaption{Листинг}

% Общие параметры листингов
\lstset{
  %frame=TB,
  showstringspaces=false,
  tabsize=4,
  basicstyle=\linespread{1.0}\tt\small, % делаем листинги компактнее
  breaklines=true,
  texcl=true, % русские буквы в комментах
  captionpos=b,
  aboveskip=\baselineskip,
  commentstyle=\tt
}

\newcommand{\todo}[1]{\textcolor{red}{#1}}

% DEBUG
% \usepackage{showframe}

\Faculty{Факультет вычислительной математики и кибернетики}
\Chair{Кафедра системного программирования}
\Course{Параллельные высокопроизводительные вычисления}
\Year{2024}
  \Date{6 Ноября}
  \City{Москва}
  \AuthorText{Автор:}
  \Author{Егоров Илья Георгиевич}
  \AuthorEng{Ilya Yegorov}
  \AcadGroup{527}

  \TitleTop{Многопоточная реализация операций с сеточными данными}
  % Раскомментируйте, если нужны еще строчки названия
  \TitleMiddle{на неструктурированной смешанной сетке,}
  \TitleBottom{решение СЛАУ}
  % Uncomment if you need English title
  % \TitleTopEng{Thesis theme, first line}
  % \TitleMiddleEng{Thesis theme, second line}
  % \TitleBottomEng{Thesis theme, third line}

\docname{ОТЧЁТ}                                      

%%%% DON'T change this. It is here because .sty does not support cyrillic cp properly %%%%
\TitlePageText{Титульная страница}
\University{Московский государственный университет имени М.В.Ломоносова}
\GrText{группа}
\ListingText{Листинг}
\AlgorithmText{Алгоритм}

% Set PDF title and author
\hypersetup{
  pdftitle={\PDFTitle},
  pdfauthor={\PDFAuthor}
}

\begin{document}
\fixmargins
 \makepreliminarypages

\oneandhalfspace

\pdfbookmark[section]{\contentsname}{toc}
\tableofcontents

\section{Описание задания и программной реализации}
\subsection{Постановка задачи}
Работа программы состоит из 4 этапов:
\begin{enumerate}
  \item \textbf{Generate}~--- генерация графа/портрета по тестовой сетке;
  \item \textbf{Fill}~--- заполнение матрицы по заданному портрету;
  \item \textbf{Solve}~--- решение СЛАУ с полученной матрицей;
  \item \textbf{Report}~--- проверка корректности программы и выдача измерений. 
\end{enumerate}

\subsubsection{Генерация портрета на основе тестовой сетки}
Будем иметь дело с двухмерной неструктурированной смешанной сеткой, состоящей из
треугольников и четырехугольников.  Решетка состоит из $N = N_x * N_y$ клеточек.
У каждой клетки есть позиция $(x, y)$, где $i$~--- номер строки в решетке,
$j$~--- номер столбца. Нумерация клеток в решетке построчно слева направо,
сверху вниз. Номер клеточки в единой общей нумерации: $I = i * N_x + j$. Далее
часть квадратных клеток делятся на треугольники следующим образом: $K_1$ и
$K_2$~--- количество идущих подряд треугольников и четырехугольников
соотвественно. На данном этапе нужно сгенерировать "топологию" связей ячеек:
построить графа, и по нему сгенерировать портрет матрицы смежности, дополнив
этот портрет главной диагональю, где вершины графа~--- элементы сетки.

\textbf{Входные данные:}
\begin{itemize}
  \item $N_x, N_y$~--- число клеток в решетке по вертикали и горизонтали;
  \item $K_1, K_2$~--- параметры для количества треугольных и четырехугольных
    элементов.
\end{itemize}

\textbf{Выходные данные:}
\begin{itemize}
  \item $N$~--- размер матрицы (число вершин в графе);
  \item $IA, JA$~--- портрет разреженной матрицы смежности графа, (в формате
    CSR). 
\end{itemize}

\subsubsection{Построение СЛАУ по заданному портрету матрицы}

\textbf{Входные данные:}
\begin{itemize}
  \item $N$~--- размер матрицы (число вершин в графе);
  \item $IA, JA$~--- портрет разреженной матрицы смежности графа, (в формате
    CSR). 
\end{itemize}

\textbf{Выходные данные:}
\begin{itemize}
  \item $A$~--- массив ненулевых коэффициентов матрицы (размера $IA[N]$);
  \item $b$~--- вектор правой части (размера $N$).
\end{itemize}

Правила заполнения:
\begin{itemize}
  \item $a_{ij} = \cos(i * j + i + j), i \neq j, j \in Col(i)$
  \item $a_{ii} = 1.234 \sum_{j, j \neq i} |a_{ij}|$
  \item $b_i = \sin(i)$
\end{itemize}

\subsubsection{Решение СЛАУ итерационным методом}
\textbf{Входные данные:}
\begin{itemize}
  \item $N$~--- размер матрицы (число вершин в графе);
  \item $IA, JA$~--- портрет разреженной матрицы смежности графа, (в формате
    CSR). 
  \item $A$~--- массив ненулевых коэффициентов матрицы (размера $IA[N]$);
  \item $b$~--- вектор правой части (размера $N$).
  \item $eps$~--- критерий остановки ($\epsilon$), которым определяется точность
    решения;
  \item $maxit$~--- максимальное число итераций.
\end{itemize}

\textbf{Выходные данные:}
\begin{itemize}
  \item $x$~--- вектор решения (размера $N$);
  \item $n$~--- количество выполненных итераций;
  \item $r$~--- L2 норма невязки (невязка – это вектор $r = Ax - b$).
\end{itemize}

Будем использовать такой алгоритм предобусловленного метода CG.

Для решателя понадобится несколько вычислительных функций для базовых операций:
\begin{itemize}
  \item Матрично-векторное произведение с разреженной матрицей (sparse
    matrix-vector);
  \item Скалярное произведение;
  \item Поэлементное сложение двух векторов с умножением одного из них на
    скаляр.
\end{itemize}

\subsubsection{Проверка корректности программы и выдача измерений}
На этом этапе нужно проверить, что невязка системы удовлетворяет заданной
точности, выдать значение фактической невязки, и распечатать табличку
таймирования, в которой указано, сколько времени в секундах затрачено на:
\begin{enumerate}
  \item Этап генерации;
  \item Этап заполнения;
  \item Этап решения СЛАУ;
  \item Каждую из трех кернел-функций, базовых алгебраических операций решателя
    СЛАУ.
\end{enumerate}

\subsection{Программная реализация}
\section{Исследование производительности}
\subsection{Характеристики вычислительной системы}
\subsection{Результаты измерений производительности}
\subsubsection{Последовательная производительность}
\subsubsection{Параллельное ускорение}
\section{Анализ полученных результатов}

\end{document}

